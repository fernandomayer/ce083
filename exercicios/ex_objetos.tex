\documentclass[a4paper,11pt,fleqn]{article}\usepackage[]{graphicx}\usepackage[]{color}
%% maxwidth is the original width if it is less than linewidth
%% otherwise use linewidth (to make sure the graphics do not exceed the margin)
\makeatletter
\def\maxwidth{ %
  \ifdim\Gin@nat@width>\linewidth
    \linewidth
  \else
    \Gin@nat@width
  \fi
}
\makeatother

\definecolor{fgcolor}{rgb}{0, 0, 0}
\newcommand{\hlnum}[1]{\textcolor[rgb]{0,0,0}{#1}}%
\newcommand{\hlstr}[1]{\textcolor[rgb]{0,0,0}{#1}}%
\newcommand{\hlcom}[1]{\textcolor[rgb]{0.4,0.4,0.4}{\textit{#1}}}%
\newcommand{\hlopt}[1]{\textcolor[rgb]{0,0,0}{\textbf{#1}}}%
\newcommand{\hlstd}[1]{\textcolor[rgb]{0,0,0}{#1}}%
\newcommand{\hlkwa}[1]{\textcolor[rgb]{0,0,0}{\textbf{#1}}}%
\newcommand{\hlkwb}[1]{\textcolor[rgb]{0,0,0}{\textbf{#1}}}%
\newcommand{\hlkwc}[1]{\textcolor[rgb]{0,0,0}{\textbf{#1}}}%
\newcommand{\hlkwd}[1]{\textcolor[rgb]{0,0,0}{\textbf{#1}}}%
\let\hlipl\hlkwb

\usepackage{framed}
\makeatletter
\newenvironment{kframe}{%
 \def\at@end@of@kframe{}%
 \ifinner\ifhmode%
  \def\at@end@of@kframe{\end{minipage}}%
  \begin{minipage}{\columnwidth}%
 \fi\fi%
 \def\FrameCommand##1{\hskip\@totalleftmargin \hskip-\fboxsep
 \colorbox{shadecolor}{##1}\hskip-\fboxsep
     % There is no \\@totalrightmargin, so:
     \hskip-\linewidth \hskip-\@totalleftmargin \hskip\columnwidth}%
 \MakeFramed {\advance\hsize-\width
   \@totalleftmargin\z@ \linewidth\hsize
   \@setminipage}}%
 {\par\unskip\endMakeFramed%
 \at@end@of@kframe}
\makeatother

\definecolor{shadecolor}{rgb}{.97, .97, .97}
\definecolor{messagecolor}{rgb}{0, 0, 0}
\definecolor{warningcolor}{rgb}{1, 0, 1}
\definecolor{errorcolor}{rgb}{1, 0, 0}
\newenvironment{knitrout}{}{} % an empty environment to be redefined in TeX

\usepackage{alltt}

%%----------------------------------------------------------------------
%% opções comuns
\usepackage[brazilian]{babel}
\usepackage[utf8]{inputenc}
\usepackage[T1]{fontenc}
\usepackage{textcomp}
%\usepackage[margin=2cm]{geometry}
\usepackage{indentfirst}
\usepackage{fancybox}
%\usepackage[usenames,dvipsnames]{color}
\usepackage{amsmath,amsfonts,amssymb,amsthm}
\usepackage{lscape}
\usepackage{natbib}
\setlength{\bibsep}{0.0pt}
\usepackage{url}
\usepackage{multicol}
\usepackage{multirow}
\usepackage[final]{pdfpages}
\usepackage{setspace}
\usepackage{paralist} % enumitem, compactitem
%%----------------------------------------------------------------------

%%----------------------------------------------------------------------
%% FLOATS: graficos e tabelas
\usepackage{graphicx}
\usepackage{float} % fornece a opção [H] para floats
\usepackage{longtable}
\usepackage{supertabular}
%% captions e headings em sans-serif
\usepackage[font={sf},labelfont={sf,bf}]{caption}
\usepackage{subcaption}
\renewcommand{\thesubfigure}{\Alph{subfigure}}
\usepackage{titlesec}
\titleformat*{\section}{\normalsize\bfseries\sffamily}
\titleformat*{\subsection}{\normalsize\bfseries\sffamily}
\titleformat*{\subsubsection}{\normalsize\bfseries\sffamily}
\titleformat*{\paragraph}{\normalsize\bfseries\sffamily}
\titleformat*{\subparagraph}{\normalsize\bfseries\sffamily}
\theoremstyle{definition}
\newtheorem*{mydef}{Definição}
%%----------------------------------------------------------------------

%%----------------------------------------------------------------------
%% definiçoes de hyperref e xcolor
\usepackage{hyperref}
\usepackage{xcolor}
%%----------------------------------------------------------------------

%%----------------------------------------------------------------------
%% FONTES

%% micro-tipografia
\usepackage[protrusion=true,expansion=true]{microtype}
%% Bitstream Charter with mathdesign
%% \usepackage{lmodern} % sans-serif: Latin Modern
%% \usepackage[charter]{mathdesign} % serif: Bitstream Charter
\usepackage[scaled]{beramono} % truetype: Bistream Vera Sans Mono
%% \usepackage[scaled]{helvet}
%\usepackage{inconsolata}
\usepackage{mathpazo}
\linespread{1.05}

%\usepackage[sf]{titlesec}
%%----------------------------------------------------------------------

%%----------------------------------------------------------------------
%% hifenização
\usepackage[htt]{hyphenat} % permite hifenizar texttt. Ao inves disso
% pode usar \allowbreak no ponto qu quiser quebrar dentro do texttt
\hyphenation{con-si-de-ra-ção pes-que-i-ros pes-que-i-ra se-gui-do-ras
  di-fe-ren-tes pla-ni-lha pla-ni-lhão re-fe-ren-te con-ta-gem
  em-bar-ques qua-li-da-de a-le-a-to-ri-za-dos}
%%----------------------------------------------------------------------

%%----------------------------------------------------------------------
%% comandos customizados
\usepackage{xspace} % lida com os espaços depois dos comandos
\providecommand{\eg}{\textit{e.g.}\xspace}
\providecommand{\ie}{\textit{i.e.}\xspace}
\providecommand{\R}{\textsf{R}\xspace}
\newcommand{\mb}[1]{\mathbf{#1}}
\newcommand{\bs}[1]{\boldsymbol{#1}}
\providecommand{\E}{\text{E}}
\providecommand{\Var}{\text{Var}}
\providecommand{\logit}{\text{logit}}
%% Para alterar o titulo do thebibliography
\addto\captionsbrazilian{%
  \renewcommand{\refname}{Bibliografia}
}
%%----------------------------------------------------------------------

%%----------------------------------------------------------------------
%% Comandos para deixar o texto mais compacto
\usepackage{marginnote}
\usepackage[top=1cm, bottom=1cm, inner=1cm, outer=1cm,nohead, nofoot, heightrounded, marginparsep=.05cm]{geometry}
\setlength{\parindent}{0pt}
%%----------------------------------------------------------------------
\IfFileExists{upquote.sty}{\usepackage{upquote}}{}
\begin{document}

\reversemarginpar % para colocar a marginnote a esquerda





\hrule
\vspace{0.3cm}
%%----------------------------------------------------------------------
%% Cabeçalho e título
% \begin{minipage}[c]{.15\textwidth}
% \flushleft
% \includegraphics[width=1.4cm]{../../../logos/furgPM.png}
% \end{minipage}
% \begin{minipage}[c]{.70\textwidth}
%   \begin{center}
%     \textsc{Universidade Federal do Rio Grande --- FURG} \\
%     \textsc{Instituto de Matemática, Estatística e Física --- IMEF} \\
%     %Prof. Fernando de Pol Mayer
%   \end{center}
% \end{minipage}\hfill
% \begin{minipage}[c]{.15\textwidth}
% \flushright
% \includegraphics[width=1.4cm]{../../../logos/logo_IMEF2.png}
% \end{minipage}

% \vspace{0.3cm}
% \hrule
% \vspace{0.3cm}


\begin{minipage}[c]{.85\textwidth}
  Estatística Computacional I --- CE083 \\
  Prof. Fernando de Pol Mayer --- Departamento de Estatística --- DEST \\
  Exercícios: comandos básicos e classes de objetos \\
  Nome:  \hfill GRR: \hspace{2cm}
\end{minipage}\hfill
\begin{minipage}[c]{.15\textwidth}
\flushright
\includegraphics[width=2.2cm]{../img/ufpr-logo2.jpg}
\end{minipage}

\vspace{0.3cm}
\hrule
\vspace{0.3cm}

%% Instruções:
%% \begin{compactitem}[---]
%% \item Responda todas as questões na folha em anexo. Apenas a resposta
%%   final deve ser à caneta.
%% %% \item Use pelo menos 3 casas decimais nas questões que envolvem
%% %%   cálculo.
%% \item As questões não precisam ser respondidas em ordem, mas não esqueça
%%   de identificá-las.
%% \end{compactitem}

%% \vspace{0.3cm}
%% \hrule
%% \vspace{0.3cm}

% \begin{compactenum}[1.]
% \item A linguagem R é um dialeto de qual das seguintes linguagens de
%   programação?
%   \begin{compactenum}
%   \item[] (a) C++ \qquad (b) Python \qquad (c) S \qquad (d) Fortran
%   \end{compactenum}
%   %% \begin{compactenum}
%   %% \item C++
%   %% \item Python
%   %% \item S
%   %% \item Fortran
%   %% \end{compactenum}
% \end{compactenum}

% \vspace{0.3cm}
% \hrule
% \vspace{0.3cm}

% \begin{compactenum}[2.]
% \item A definição de \textit{software livre} consiste de quatro
%   liberdades. Qual das frases abaixo \underline{não} é uma destas
%   liberdades que fazem parte dessa definição?
%   \begin{compactenum}
%   \item A liberdade de restringir o acesso ao código fonte do programa
%   \item A liberdade de estudar como o programa funciona, e adaptá-lo
%     para as suas necessidades
%   \item A liberdade de aprimorar o programa, e distribuir suas
%     modificações para que todos se beneficiem
%   \item A liberdade de redistribuir cópias do programa
%   \end{compactenum}
% \end{compactenum}

% \vspace{0.3cm}
% \hrule
% \vspace{0.3cm}

\begin{compactenum}[1.]
\item Ao executar a expressão
\begin{knitrout}\small
\definecolor{shadecolor}{rgb}{1, 1, 1}\color{fgcolor}\begin{kframe}
\begin{alltt}
\hlstd{x} \hlkwb{<-} \hlkwd{c}\hlstd{(}\hlnum{4}\hlstd{,} \hlnum{7}\hlstd{,} \hlnum{10}\hlstd{,} \hlnum{1}\hlstd{)}
\end{alltt}
\end{kframe}
\end{knitrout}
qual a classe do objeto \texttt{x}, determinado pela função \texttt{class()}?
\begin{compactenum}
  \item[] (a) \texttt{integer} \qquad (b) \texttt{matrix} \qquad (c)
    \texttt{numeric} \qquad (d) \texttt{complex}
\end{compactenum}

\end{compactenum}

\vspace{0.3cm}
\hrule
\vspace{0.3cm}

\begin{compactenum}[2.]
\item Qual é a classe do objeto definido pela expressão
\begin{knitrout}\small
\definecolor{shadecolor}{rgb}{1, 1, 1}\color{fgcolor}\begin{kframe}
\begin{alltt}
\hlstd{x} \hlkwb{<-} \hlkwd{c}\hlstd{(}\hlnum{4}\hlstd{,} \hlstr{"a"}\hlstd{,} \hlnum{TRUE}\hlstd{)}
\end{alltt}
\end{kframe}
\end{knitrout}
\begin{compactenum}
  \item[] (a) \texttt{logical} \qquad (b) \texttt{character} \qquad (c)
    \texttt{numeric} \qquad (d) \texttt{factor}
\end{compactenum}

\end{compactenum}

\vspace{0.3cm}
\hrule
\vspace{0.3cm}

\begin{compactenum}[3.]
\item Qual o resultado do comando abaixo?
\begin{knitrout}\small
\definecolor{shadecolor}{rgb}{1, 1, 1}\color{fgcolor}\begin{kframe}
\begin{alltt}
\hlstd{> }\hlkwd{rep}\hlstd{(}\hlkwd{c}\hlstd{(}\hlstr{"A"}\hlstd{,} \hlstr{"B"}\hlstd{,} \hlstr{"C"}\hlstd{),} \hlkwc{times} \hlstd{=} \hlkwd{c}\hlstd{(}\hlnum{1}\hlstd{,} \hlnum{3}\hlstd{,} \hlnum{2}\hlstd{))}
\end{alltt}
\end{kframe}
\end{knitrout}
\begin{tabular}{| p{1cm} | p{1cm} | p{1cm} | p{1cm} | p{1cm} | p{1cm}
  | p{1cm} | p{1cm} | p{1cm} | p{1cm} |}
  \hline
  & & & & & & & & & \\
  \hline
\end{tabular}
\end{compactenum}

\vspace{0.3cm}
\hrule
\vspace{0.3cm}

\begin{compactenum}[4.]
\item Qual é a classe do objeto definido pela expressão
\begin{knitrout}\small
\definecolor{shadecolor}{rgb}{1, 1, 1}\color{fgcolor}\begin{kframe}
\begin{alltt}
\hlstd{x} \hlkwb{<-} \hlkwd{c}\hlstd{(}\hlnum{4}\hlstd{,} \hlnum{TRUE}\hlstd{)}
\end{alltt}
\end{kframe}
\end{knitrout}
\begin{compactenum}
  \item[] (a) \texttt{numeric} \qquad (b) \texttt{logical} \qquad (c)
    \texttt{character} \qquad (d) \texttt{vector}
\end{compactenum}

\end{compactenum}

\vspace{0.3cm}
\hrule
\vspace{0.3cm}

\begin{compactenum}[5.]
\item Considere os dois vetores abaixo
\begin{knitrout}\small
\definecolor{shadecolor}{rgb}{1, 1, 1}\color{fgcolor}\begin{kframe}
\begin{alltt}
\hlstd{x} \hlkwb{<-} \hlkwd{c}\hlstd{(}\hlnum{1}\hlstd{,} \hlnum{3}\hlstd{,} \hlnum{5}\hlstd{)}
\hlstd{y} \hlkwb{<-} \hlkwd{c}\hlstd{(}\hlnum{3}\hlstd{,} \hlnum{2}\hlstd{,} \hlnum{10}\hlstd{)}
\end{alltt}
\end{kframe}
\end{knitrout}
Qual o resultado da expressão
\begin{knitrout}\small
\definecolor{shadecolor}{rgb}{1, 1, 1}\color{fgcolor}\begin{kframe}
\begin{alltt}
\hlkwd{rbind}\hlstd{(x, y)}
\end{alltt}
\end{kframe}
\end{knitrout}
\begin{compactenum}
  \item Uma matriz 2x2
  \item Uma matriz 2x3
  \item Um vetor de comprimento 6
  \item Um \texttt{data.frame} com 2 linhas e 3 colunas
\end{compactenum}

\end{compactenum}

\vspace{0.3cm}
\hrule
\vspace{0.3cm}

\begin{compactenum}[6.]
\item Considere os dois vetores abaixo
\begin{knitrout}\small
\definecolor{shadecolor}{rgb}{1, 1, 1}\color{fgcolor}\begin{kframe}
\begin{alltt}
\hlstd{x} \hlkwb{<-} \hlkwd{c}\hlstd{(}\hlnum{5}\hlstd{,} \hlnum{10}\hlstd{)}
\hlstd{y} \hlkwb{<-} \hlkwd{c}\hlstd{(}\hlnum{2}\hlstd{,} \hlnum{8}\hlstd{)}
\end{alltt}
\end{kframe}
\end{knitrout}
Qual o resultado da expressão
\begin{knitrout}\small
\definecolor{shadecolor}{rgb}{1, 1, 1}\color{fgcolor}\begin{kframe}
\begin{alltt}
\hlkwd{cbind}\hlstd{(x, y)}
\end{alltt}
\end{kframe}
\end{knitrout}
\begin{compactenum}
  \item Uma matriz 2x1
  \item Uma matriz 2x2
  \item Um vetor de comprimento 4
  \item Um \texttt{data.frame} com 2 linhas e 2 colunas
\end{compactenum}

\end{compactenum}

\vspace{0.3cm}
\hrule
\vspace{0.3cm}

\begin{compactenum}[7.]
\item Uma propriedade fundamental de vetores no R é que
\begin{compactenum}
  \item elementos de um vetor podem ser de classes diferentes
  \item um vetor não possui atributos como \texttt{dim()} e
    \texttt{length()}
  \item um vetor não pode ser convertido para outras classes
  \item elementos de um vetor devem ser todos da mesma classe
\end{compactenum}
\end{compactenum}

\vspace{0.3cm}
\hrule
\vspace{0.3cm}

\newpage

\vspace{0.3cm}
\hrule
\vspace{0.3cm}

\begin{compactenum}[8.]
\item Considere o objeto abaixo:
\begin{knitrout}\small
\definecolor{shadecolor}{rgb}{1, 1, 1}\color{fgcolor}\begin{kframe}
\begin{alltt}
\hlstd{x} \hlkwb{<-} \hlkwd{list}\hlstd{(}\hlnum{2}\hlstd{,} \hlstr{"a"}\hlstd{,} \hlstr{"b"}\hlstd{,} \hlnum{TRUE}\hlstd{)}
\end{alltt}
\end{kframe}
\end{knitrout}
Qual o comprimento e a classe do objeto \texttt{x}, respectivamente?
\begin{compactenum}
  \item 2, \quad \texttt{logical}
  \item 4, \quad \texttt{character}
  \item 4, \quad \texttt{list}
  \item 2, \quad \texttt{numeric}
\end{compactenum}

\end{compactenum}

\vspace{0.3cm}
\hrule
\vspace{0.3cm}

\begin{compactenum}[9.]
\item Considere os dois vetores abaixo:
\begin{knitrout}\small
\definecolor{shadecolor}{rgb}{1, 1, 1}\color{fgcolor}\begin{kframe}
\begin{alltt}
\hlstd{x} \hlkwb{<-} \hlnum{1}\hlopt{:}\hlnum{4}
\hlstd{y} \hlkwb{<-} \hlnum{2}\hlopt{:}\hlnum{3}
\end{alltt}
\end{kframe}
\end{knitrout}
Qual o resultado da expressão
\begin{knitrout}\small
\definecolor{shadecolor}{rgb}{1, 1, 1}\color{fgcolor}\begin{kframe}
\begin{alltt}
\hlstd{x} \hlopt{*} \hlstd{y}
\end{alltt}
\end{kframe}
\end{knitrout}
\begin{tabular}{| p{1cm} | p{1cm} | p{1cm} | p{1cm} | p{1cm} | p{1cm}
  | p{1cm} | p{1cm} | p{1cm} | p{1cm} |}
  \hline
  & & & & & & & & & \\
  \hline
\end{tabular}

\end{compactenum}

\vspace{0.3cm}
\hrule
\vspace{0.3cm}

\begin{compactenum}[10.]
\item Considere os dois vetores abaixo:
\begin{knitrout}\small
\definecolor{shadecolor}{rgb}{1, 1, 1}\color{fgcolor}\begin{kframe}
\begin{alltt}
\hlstd{x} \hlkwb{<-} \hlnum{1}\hlopt{:}\hlnum{4}
\hlstd{y} \hlkwb{<-} \hlnum{2}\hlopt{:}\hlnum{4}
\end{alltt}
\end{kframe}
\end{knitrout}
Qual o resultado da expressão
\begin{knitrout}\small
\definecolor{shadecolor}{rgb}{1, 1, 1}\color{fgcolor}\begin{kframe}
\begin{alltt}
\hlstd{x} \hlopt{+} \hlstd{y}
\end{alltt}
\end{kframe}
\end{knitrout}
\begin{tabular}{| p{1cm} | p{1cm} | p{1cm} | p{1cm} | p{1cm} | p{1cm}
  | p{1cm} | p{1cm} | p{1cm} | p{1cm} |}
  \hline
  & & & & & & & & & \\
  \hline
\end{tabular}

\end{compactenum}

\vspace{0.3cm}
\hrule
\vspace{0.3cm}

\begin{compactenum}[11.]
\item Considere o objeto abaixo:

\begin{knitrout}\small
\definecolor{shadecolor}{rgb}{1, 1, 1}\color{fgcolor}\begin{kframe}
\begin{alltt}
\hlstd{> }\hlstd{dados}
\end{alltt}
\begin{verbatim}
   Ozonio Rad.Solar Vento Temp Mes   Estacao
1      41       190   7.4   67   1     Verao
2      36       118   8.0   72   2     Verao
3      12       149  12.6   74   3     Verao
4      18       313  11.5   62   4    Outono
5      NA        NA  14.3   56   5    Outono
6      28        NA  14.9   66   6    Outono
7      23       299   8.6   65   7   Inverno
8      19        99  13.8   59   8   Inverno
9       8        19  20.1   61   9   Inverno
10     NA       194   8.6   69  10 Primavera
11      7        NA   6.9   74  11 Primavera
12     16       256   9.7   69  12 Primavera
\end{verbatim}
\end{kframe}
\end{knitrout}
\begin{compactenum}
  \item Qual a classe do objeto \texttt{dados}? \newline
  \item Qual a classe da coluna \texttt{Ozonio}? \newline
  \item Qual a classe da coluna \texttt{Mes}? \newline
  \item Qual a classe da coluna \texttt{Estacao}? \newline
\end{compactenum}
\end{compactenum}

\vspace{0.3cm}
\hrule
\vspace{0.3cm}

\begin{compactenum}[12.]
\item Na questão anterior, a coluna \texttt{Ozonio} é composta pelos
  seguintes valores
\begin{knitrout}\small
\definecolor{shadecolor}{rgb}{1, 1, 1}\color{fgcolor}\begin{kframe}
\begin{verbatim}
 [1] 41 36 12 18 NA 28 23 19  8 NA  7 16
\end{verbatim}
\end{kframe}
\end{knitrout}
O que signifca o termo \texttt{NA}, e o que ele está representando neste
contexto? \newline \newline \newline
\end{compactenum}

\vspace{0.3cm}
\hrule
\vspace{0.3cm}

\newpage

\vspace{0.3cm}
\hrule
\vspace{0.3cm}


\begin{compactenum}[13.]
\item Na questão anterior, considerando o objeto \texttt{dados}, se
  utilizarmos a expressão condicional abaixo na coluna com o nome
  \texttt{Rad.Solar}:
\begin{knitrout}\small
\definecolor{shadecolor}{rgb}{1, 1, 1}\color{fgcolor}\begin{kframe}
\begin{alltt}
\hlstd{> }\hlstd{Rad.Solar} \hlopt{>=} \hlnum{180}
\end{alltt}
\end{kframe}
\end{knitrout}
Qual seria o resultado?
\begin{compactenum}
  \item \texttt{FALSE TRUE FALSE  TRUE    NA    NA  TRUE FALSE FALSE
      TRUE    NA  FALSE}
  \item \texttt{TRUE FALSE FALSE  TRUE    TRUE    TRUE  TRUE FALSE FALSE
      TRUE    TRUE  TRUE}
  \item \texttt{TRUE FALSE FALSE  TRUE    FALSE    FALSE  TRUE FALSE
      FALSE  TRUE    FALSE  TRUE}
  \item \texttt{TRUE FALSE FALSE  TRUE    NA    NA  TRUE FALSE FALSE
      TRUE    NA  TRUE}
\end{compactenum}

\end{compactenum}

\vspace{0.3cm}
\hrule
\vspace{0.3cm}

\begin{compactenum}[14.]
\item O resultado da função \texttt{str()} aplicada a um objeto chamado
  \texttt{dados2} gerou o seguinte resultado:

\begin{knitrout}\small
\definecolor{shadecolor}{rgb}{1, 1, 1}\color{fgcolor}\begin{kframe}
\begin{alltt}
\hlstd{> }\hlkwd{str}\hlstd{(dados2)}
\end{alltt}
\begin{verbatim}
'data.frame':	20 obs. of  2 variables:
 $ Racao     : Factor w/ 4 levels "A","B","C","D": 1 1 1 1 1 2 2 2 2 2 ...
 $ Ganho.Peso: int  35 19 31 15 30 40 35 46 41 33 ...
\end{verbatim}
\end{kframe}
\end{knitrout}
Com isso, responda:
\begin{compactenum}
  \item Qual a classe do objeto \texttt{dados2}? \newline
  \item Quantas linhas e colunas possui esse objeto? \newline
  \item Qual o nome das colunas desse objeto? \newline
  \item Qual a classe de cada uma das colunas? \newline
\end{compactenum}
\end{compactenum}

\vspace{0.3cm}
\hrule
\vspace{0.3cm}

\begin{compactenum}[15.]
\item Para gerar 10 números aleatórios de uma distribuição uniforme,
  $\text{U}[5,15]$, usamos a função \texttt{runif()} que possui os
  argumentos: \texttt{n}, \texttt{min}, e \texttt{max} (nessa
  ordem). Indique qual das alternativas está \underline{errada}:
  \begin{compactenum}
      \item \texttt{runif(5, n = 10, max = 15)}
      \item \texttt{runif(10, 5, 15)}
      \item \texttt{runif(min = 5, n = 10, max = 15)}
      \item \texttt{runif(n = 10, 15, 5)}
  \end{compactenum}
\end{compactenum}

\vspace{0.3cm}
\hrule
\vspace{0.3cm}

\begin{compactenum}[16.]
\item Qual o resultado do comando abaixo?
\begin{knitrout}\small
\definecolor{shadecolor}{rgb}{1, 1, 1}\color{fgcolor}\begin{kframe}
\begin{alltt}
\hlstd{> }\hlkwd{seq}\hlstd{(}\hlkwc{from} \hlstd{=} \hlnum{1}\hlstd{,} \hlkwc{to} \hlstd{=} \hlnum{20}\hlstd{,} \hlkwc{by} \hlstd{=} \hlnum{4}\hlstd{)}
\end{alltt}
\end{kframe}
\end{knitrout}
\begin{tabular}{| p{1cm} | p{1cm} | p{1cm} | p{1cm} | p{1cm} | p{1cm}
  | p{1cm} | p{1cm} | p{1cm} | p{1cm} |}
  \hline
  & & & & & & & & & \\
  \hline
\end{tabular}
\end{compactenum}

\vspace{0.3cm}
\hrule
\vspace{0.3cm}

\begin{compactenum}[17.]
\item Considere o objeto abaixo:
\begin{knitrout}\small
\definecolor{shadecolor}{rgb}{1, 1, 1}\color{fgcolor}\begin{kframe}
\begin{alltt}
\hlstd{> }\hlstd{epoca} \hlkwb{<-} \hlkwd{factor}\hlstd{(}\hlkwd{c}\hlstd{(}\hlstr{"Verao"}\hlstd{,} \hlnum{NA}\hlstd{,} \hlstr{"Verao"}\hlstd{,} \hlnum{NA}\hlstd{,} \hlstr{"Outono"}\hlstd{,} \hlstr{"Outono"}\hlstd{,}
\hlstd{  }                  \hlstr{"Inverno"}\hlstd{,} \hlstr{"Inverno"}\hlstd{,} \hlnum{NA}\hlstd{,} \hlnum{NA}\hlstd{,} \hlstr{"Primavera"}\hlstd{,} \hlnum{NA}\hlstd{))}
\end{alltt}
\end{kframe}
\end{knitrout}
Qual é a ordem padrão dos níveis deste fator?
\begin{compactenum}
\item \texttt{Levels: Inverno Primavera Outono Verao}
\item \texttt{Levels: Verao Outono Inverno Primavera NA}
\item \texttt{Levels: Inverno Outono Primavera Verao NA}
\item \texttt{Levels: Inverno Outono Primavera Verao}
\end{compactenum}
\end{compactenum}

\vspace{0.3cm}
\hrule
\vspace{0.3cm}

\newpage

\vspace{0.3cm}
\hrule
\vspace{0.3cm}

\begin{compactenum}[18.]
\item Para criar a matriz abaixo:
    \[ \left[ \begin{array}{cc}
        4 & 1 \\
        9 & 5 \\
        10 & 7
      \end{array} \right] \]
Qual seria o comando correto?
    \begin{compactenum}
    \item \texttt{matrix(c(4, 1, 9, 5, 10, 7), nrow = 3)}
    \item \texttt{matrix(c(4, 9, 10, 1, 5, 7), ncol = 2, byrow = TRUE)}
    \item \texttt{matrix(c(4, 9, 10, 1, 5, 7), nrow = 2)}
    \item \texttt{matrix(c(4, 1, 9, 5, 10, 7), ncol = 2, byrow = TRUE)}
    \end{compactenum}
\end{compactenum}

\vspace{0.3cm}
\hrule
\vspace{0.3cm}


\end{document}
